\documentclass[11pt,extrafontsizes,twoside,openany,final]{memoir}

% THIS DOCUMENT REQUIRES XeLaTeX TO COMPILE!

\usepackage{lettrine}		% Used for the fancy caps at the start of each chapter
\usepackage{xspace}		    % Takes care of spaces after macros
\usepackage{amsmath}		% Provides the align environment, used in chapter 13 for the notes
\usepackage[protrusion=true]{microtype}
                            % Makes right margin neater by letting certain punctuation protrude slightly
\usepackage{fontspec}		% For the many fonts
\usepackage{xunicode}       % Probably needed for something
\usepackage{xstring}        % Needed for something
\usepackage{xfrac}          % No idea
\usepackage{censor}         % Used in "Time Pressure" to redact Hermione's name
\usepackage[useregional]{datetime2}
                            % Used for the build timestamp in the colophon

\usepackage[bookmarks=true,unicode=true,pdfborder={0 0 0},
	pdftitle={Harry Potter and the Methods of Rationality},
	pdfauthor={LessWrong}, breaklinks={true},
	pdfkeywords={Harry Potter, rationality},pdfencoding=auto
]{hyperref}                 % Hyperlinks and bookmarks in PDF

\usepackage[autostyle=false, style=english]{csquotes}
\MakeOuterQuote{"}	        % Automatic curly quotes

\usepackage{ucntn}          % Provides \NUMTONAME for all caps output
\usepackage{graphicx}       % for \reflectbox in header

%
% Set-up page sizes
%
\setstocksize{9in}{6in}
\settrimmedsize{\stockheight}{\stockwidth}{*}
\settypeblocksize{\topskip + 36\baselineskip}{4.6in}{*}
\setlrmargins{*}{*}{0.8}
\setulmargins{*}{*}{0.8}
\setheadfoot{\topskip + \baselineskip}{2\baselineskip}
\checkandfixthelayout[fixed]
\fixdvipslayout % fix for xelatex

%
% Fonts used generally (specific fonts used only once or twice are not here).
%
\setmainfont[
, Ligatures={Common,TeX}
, Extension=.ttf
, UprightFont=*-Regular
, ItalicFont=*-Italic
, BoldFont=*-Medium
, BoldItalicFont=*-Medium-Italic
]{GaramondNo8}
%
% Chapters, headings, etc.: Lumos (extra spacing looks much better)
\newfontface\hp[ExternalLocation, LetterSpace=18.0, WordSpace=1.5]{Lumos}
\newcommand{\lumos}[1]{{\hp\MakeUppercase{#1}}}
% Chapter numbers: Lumos (extreme spacing for effect)
\newfontface\hpchap[ExternalLocation, LetterSpace=96.0, WordSpace=2.5]{Lumos}
% Lumos has no ligatures
%
% Parseltongue: Alegreya Sans Light Italic
\newfontface\abysmal[ExternalLocation, Ligatures={Common,TeX}]{AlegreyaSans-LightItalic}
\newcommand\parsel[1]{{\abysmal #1}}
%
% Draws a “magic star”, used as a decoration
\newcommand{\Star}{{\fontspec[ExternalLocation]{Miscelanea.ttf}*}}
%
% Draws three “magic stars”, used as a decoration everywhere
\def\Stars{{\large\Star\kern-.6ex\lower1.3ex\hbox{\large\Star}\kern-.1ex\raise.2ex\hbox{\tiny\Star}\spacefactor1000}}

% \sbreak makes the text break with centered stars in it, used as a separator in many chapters
\def\sbreakit{\leavevmode\unskip\unskip\unskip\unskip\unskip
	\mbox{}\nobreak\hfill\mbox{}\allowbreak\rule{.60\textwidth}{.0pt}\par%
	\vskip 0pt plus 1\baselineskip\noindent{%
		\parbox[c][0pt][c]{\textwidth}{%
			\hfil \hbox{\lower14pt\hbox{\normalsize\Stars}}%
		}%
	}%
	\vskip 0pt plus 1\baselineskip%
	\par\rule{.5\textwidth}{.0pt}\vskip1pt\noindent}
  
\def\sbreak{\unskip\unskip\unskip\unskip\unskip
	\mbox{}\nobreak\hfill\mbox{}\allowbreak\rule{.60\textwidth}{.0pt}\par%
	\vskip 0pt plus 1\baselineskip\noindent{%
		\parbox[c][0pt][c]{\textwidth}{%
			\hfil \hbox{\lower14pt\hbox{\normalsize\Stars}}%
		}%
	}%
	\vskip 0pt plus 1\baselineskip%
	\par\rule{.5\textwidth}{.0pt}\vskip1pt\noindent}



%
% Custom chapter style: confusing
%
\makechapterstyle{evans}{%
	\renewcommand*{\chapnamefont}{\hpchap\normalsize}
	\renewcommand*{\chapnumfont}{\chapnamefont\normalsize}
	\renewcommand*{\chaptitlefont}{\hp\Large}
	
	\setlength{\beforechapskip}{0pt}
	\setlength{\midchapskip}{0pt}
	\setlength{\afterchapskip}{1\baselineskip}
	
	\renewcommand*{\printchapternum}{% " C H A P T E R   O N E "
		\begin{center} \chapnumfont \hyperref[contents]{CHAPTER \NUMTONAME{\thechapter}\end{center}}}

	\renewcommand*{\printchaptername}{}% Not needed, "CHAPTER" added above
	
	\renewcommand*{\printchaptertitle}[1]{% "A DAY OF VERY LOW PROBABILITY"
		\vskip 1cm \begin{center}\chaptitlefont \MakeUppercase{##1}\end{center}\par \vskip 1cm}
	
	\renewcommand*{\chaptermark}[1]{% Don't know what this affects
		\markboth{\MakeUppercase{##1}}{%
		  \MakeUppercase{\chaptername}~\NUMTONAME{\thechapter}}}

	\renewcommand*{\tocmark}{\markboth{}{\MakeUppercase{Contents}}}
	
	\renewcommand{\tocheadstart}{\chapterheadstart}
	\renewcommand{\aftertoctitle}{\thispagestyle{empty}\afterchaptertitle}
}
\chapterstyle{evans}% Actually implements the above chapter style code

%
% Subsection
%
\setsubsecheadstyle{\scshape}
\beforesubsecskip=1.5\baselineskip % skip before the heading
\aftersubsecskip=.5\baselineskip plus .5\baselineskip % skip after the heading
\setsubsechook{\nopagebreak\vskip 0pt plus 3\baselineskip}

%
% Configure line spacing and paragraph spacing, and modifying penalties
%
\linespread{1.15}
\raggedbottom
%
\setlength{\emergencystretch}{.06\textwidth}
%
\clubpenalty=80
\widowpenalty=400
\brokenpenalty=10000

%
% Adjust space around lists
%
\setlength{\topsep}{.5\baselineskip plus 1\baselineskip minus .5\baselineskip}
\setlength{\partopsep}{.5\baselineskip plus 1\baselineskip minus .5\baselineskip}

%
% Lettrine font pick
%
\renewcommand{\LettrineFontHook}{\hp}
\renewcommand{\LettrineTextFont}{}
%
\newcommand\lettrinemph[3][]{\lettrine[#1]{#2}{\emph{#3}}}
% \setcounter{DefaultLines}{1}
\renewcommand{\DefaultLoversize}{0.4}
\renewcommand{\DefaultLraise}{0}

%
% Epigraph configuration (not currently used, but still here just in case)
%
\setlength{\epigraphwidth}{\textwidth}
%
\epigraphtextposition{flushleftright}
\epigraphfontsize{\footnotesize}
\setlength{\epigraphrule}{0pt}
\setlength{\beforeepigraphskip}{0pt}
\setlength{\afterepigraphskip}{\baselineskip}
%
\renewcommand{\epigraph}[2]{%
	\vspace{\beforeepigraphskip}%
	{%
		\epigraphsize%
		\begin{\epigraphflush}%
			\begin{minipage}{\epigraphwidth}%
				\centering\emph{#1}%
			\end{minipage}%
		\end{\epigraphflush}%
	}%
	\mbox{}\sbreak%
}

%
% Page numbering, footer, header (confusing)
%
\def\pageInFooter{{\small\Star\ \makebox[2em][c]{\thepage\,}\Star}}
\makeevenfoot{plain}{}{\pageInFooter}{}
\makeoddfoot{plain}{}{\pageInFooter}{}
\makeevenfoot{headings}{}{\pageInFooter}{}
\makeoddfoot{headings}{}{\pageInFooter}{}
%
\makeevenhead{headings}{\Stars}{
	\hp\hyperref[contents]{\rightmark}}{\reflectbox{\Stars}}
\makeatletter
\makeoddhead{headings}{\Stars}{\parbox{97mm}{\centering\hp\leftmark}}{\reflectbox{\Stars}}
\copypagestyle{cleared}{empty}
\makepsmarks{headings}{%
\createmark{chapter}{right}{shownumber}{\@chapapp\ }{. \ }}
\makeatother

%
% XeTeX character classes (hacks)
%
\newXeTeXintercharclass \punctuationClass
\XeTeXcharclass `\’ \punctuationClass
\XeTeXcharclass `\‘ \punctuationClass
\XeTeXcharclass `\“ \punctuationClass
\XeTeXcharclass `\” \punctuationClass
\XeTeXcharclass `\. \punctuationClass
\XeTeXcharclass `\, \punctuationClass
\XeTeXcharclass `\: \punctuationClass
\XeTeXcharclass `\? \punctuationClass
\XeTeXcharclass `\! \punctuationClass
\XeTeXcharclass `\: \punctuationClass
%
\newXeTeXintercharclass \digitClass
\XeTeXcharclass `\0 \digitClass
\XeTeXcharclass `\1 \digitClass
\XeTeXcharclass `\2 \digitClass
\XeTeXcharclass `\3 \digitClass
\XeTeXcharclass `\4 \digitClass
\XeTeXcharclass `\5 \digitClass
\XeTeXcharclass `\6 \digitClass
\XeTeXcharclass `\7 \digitClass
\XeTeXcharclass `\8 \digitClass
\XeTeXcharclass `\9 \digitClass
%
\newXeTeXintercharclass \dashClass
\XeTeXcharclass `\— \dashClass % em
\XeTeXcharclass `\– \dashClass % en
%
\newXeTeXintercharclass \hyphenClass
\XeTeXcharclass `\- \hyphenClass % hyphen
%
\XeTeXinterchartokenstate = 1 % enable functionality
%
\def\morhyphenpenalty{75}
\exhyphenpenalty=10000
% Allow linebreaks after hyphens, except when followed by punctuation
\XeTeXinterchartoks \hyphenClass 0 = {\hskip 0pt\penalty \morhyphenpenalty}
% Automatic spacing around em dashes
\XeTeXinterchartoks \dashClass 0 = {\,}
\XeTeXinterchartoks 0 \dashClass = {\,}
\XeTeXinterchartoks \dashClass 255 = {\,}
\XeTeXinterchartoks 255 \dashClass = {\,}

%
% Style definitions and hacks
%

% In the text, em dashes are encoded explicitly
% and the ellipsis is encoded with {\el}
\newcommand{\el}{\textellipsis} % Ellipsis definition
%
% Logical formatting macros
\newcommand{\shout}[1]{{\scshape #1}} % Shout means small caps
\newcommand{\scream}[1]{\MakeUppercase{#1}} % Scream means all caps
\newcommand{\abbrev}[1]{{\scshape \MakeLowercase{#1}}} % All small caps
\newcommand{\prophecy}[1]{\textit{\shout{#1}}} % Prophecies are set in italic small caps
%
\newcommand{\headline}[1]{\begin{center}\scshape #1\end{center}} % Newspaper headlines, centered
\newcommand{\inlineheadline}[1]{{\scshape #1}} % Newspaper headlines inline with text
%
\newcommand{\superscript}[1]{\ensuremath{^{\textrm{#1}}}} % Not sure if used
\newcommand{\subscript}[1]{\ensuremath{_{\textrm{#1}}}} %   Not sure if used
%
\newcommand{\accronym}[1]{#1} % Obsolete, defined to do nothing
\newcommand{\emcap}[1]{#1} % Also obsolete, defined to do nothing
%
% Macros to format certain words and abbreviations automatically
\newcommand{\AM}{{\scshape am}\xspace} % AM in small caps, for time of day
\newcommand{\PM}{{\scshape pm}\xspace} % PM in small caps, for time of day
\newcommand{\St}[0]{{st}\xspace} % May or may not be used, not sure
\newcommand{\Nd}[0]{{nd}\xspace} % May or may not be used, not sure
\newcommand{\Rd}[0]{{rd}\xspace} % May or may not be used, not sure
\newcommand{\Th}[0]{{th}\xspace} % May or may not be used, not sure
\newcommand{\SPHEW}{{\abbrev{SPHEW}}\xspace} % All small caps SPHEW
%
% No idea what the following command is for
\def\tq{{\small\raise1ex\hbox{3}\normalsize\kern-.2ex/\small\kern-.45ex\lower.1ex\hbox{4}\normalsize\spacefactor1000 }\xspace}% 
%
% Makes the written notes in italics (first used, Hogwarts acceptance letter)
\newenvironment{writtenNote}{%\parindent=1cm%
	\vskip 1\baselineskip plus 1\baselineskip minus 1\baselineskip%
	\begin{adjustwidth}{\parindent}{\parindent}%
	\par\noindent\itshape}
	{\end{adjustwidth}\vskip 1\baselineskip plus 1\baselineskip minus 1\baselineskip}
% Makes the letter address and closing
\newcommand{\letterAddress}[1]{\pagebreak[1]\noindent{}#1\nopagebreak[4]\par}
\newcommand{\letterClosing}[2][\vskip 1\baselineskip]{\nopagebreak[4]#1\par\nopagebreak[5]\noindent#2}
%
% Makes the inscriptions in small caps (first used, Gringotts)
\newenvironment{inscription}{%\parindent=1cm%
	\vskip 1\baselineskip plus 1\baselineskip minus 1\baselineskip%
	\begin{adjustwidth}{4\parindent}{\parindent}\scshape}
	{\end{adjustwidth}\vskip 1\baselineskip plus 1\baselineskip minus 1\baselineskip}
%
% Wrapper for headlines, currently defined to do nothing
\newenvironment{headlines}{}{}
%
% Not sure what these are included for:
\fboxrule=1pt
\fboxsep=-1pt
%
% Whiteboard in chapter 15
\newcommand{\hackChXV}[1]{
\vskip 0pt plus .5cm
\begin{center}
\Large
\fontspec[ExternalLocation]{Whiteboard}
\MakeUppercase{#1}
\settowidth{\versewidth}{\Large \MakeUppercase{Transfiguration is not permanent!}}
\vskip -1ex
\addfontfeature{}
\resizebox{\versewidth}{.6ex}{\rotatebox{90}{I}}
\end{center}
\vskip 0pt plus .5cm
}

%
% SPECIAL CHAPTER TITLES
% \partschapter{The Stanford Prison Experiment}{TSPE}{XIII}{Aftermaths}
% TOC: TSPE part XIII: Aftermaths
% Page header: The Stanford Prison Experiment XIII: \\? Aftermaths
% Title: The Stanford Prison Experiment, Part XIII: \\? Aftermaths
\newcommand{\namedpartchapter}[4]{%
	\chapter[%
			\texorpdfstring{%
				\abbrev{#2, part #3}: #4}{%
				#2, part #3: #4}][%
			\mbox{#1 #3:} \mbox{#4}]{%
			#1, Part~#3:\protect\linebreak[1] #4}%
}

\newcommand{\partchapter}[2]{%
	\chapter[\texorpdfstring{#1, \abbrev{part #2}}{#1, part #2}]%
		[#1 #2]{#1, Part~#2}}
		
\newcommand{\partschapter}[2]{\chapter[#1, \texorpdfstring{\abbrev{parts #2}}{parts #2}][#1 #2]{#1, Parts~#2}}

%
% Original maintainer wrote:
%     In cases where the original text didn’t line-break nicely, I did (minimal) rewordings. This
%     macro was used to mark them. The first argument is the original text (ignored), and the
%     second holds the replacement.
%
\newcommand{\replacement}[2]{#2}
\newcommand{\splitment}[3]{\discretionary{#1}{#2}{#3}}

%
% Hyphenation corrections
%
\hyphenation{Her-mi-o-ne Gran-ger bru-shes Gryf-fin-dor Le-strange 
some-where which-ev-er Hog-warts re-pli-cat-ed ran-dom sta-tis-ti-cal 
Wi-zen-gam-ot an-aly-se an-aly-sis remem-ber McGon-a-gall fun-da-men-tal}

